\documentclass{beamer}	
\mode<presentation>
 
\usepackage{pdfpages}
\usepackage{fancyvrb}
\usepackage{chemarr}

\usepackage{amsmath}		%% mathematics typesetting
\usepackage{amssymb}
 
\usepackage{epigraph}   %% nice setting of quotations

\usepackage{tabularx} %% allows to use row colours in tables

\usepackage{ulem}

\usepackage{booktabs}

\usepackage{siunitx} %% tpyeset SI units

\usepackage{CJKutf8} %% typeset Chinese characters

\usepackage{pdfpages}%% include pdfs


\usepackage{animate} %% show animated gifs

\DeclareMathAlphabet{\mathcalligra}{T1}{calligra}{m}{n}


% Color and Theme. Can be changed. However, this one's quite nice.
\usetheme{Madrid}
\definecolor{theme}{rgb}{0.84,0,0.21}
\usecolortheme[named=theme]{structure}


%%  Title information
\title[M11.10.2 Sexualentwicklung I]{M11.10.2 Sexualentwicklung und Reproduktionsphysiologie I}
\author[melanie.stefan@medicalschool-berlin.de]{}
\institute[]{Prof. Melanie Stefan - melanie.stefan@medcialschool-berlin.de}
\date{SoSe 2022}
 

% Table of contents to pop up at the beginning of each section
\AtBeginSection[]
{
  \begin{frame}<beamer>
    \frametitle{Outline}
    \tableofcontents[currentsection,currentsubsection]
  \end{frame}
}
 
\beamertemplatenavigationsymbolsempty

\begin{document}


{
  \usebackgroundtemplate{\includegraphics[width=1.2\paperwidth]{/home/melanie/Work/pictures/teaching/MSB_Titelseite.pdf}} 
\begin{frame}

 \maketitle 

$\,$\\[6cm]


\end{frame} 
}



%% Hook: 
{
  \usebackgroundtemplate{\includegraphics[width=1.2\paperwidth]{/home/melanie/Work/pictures/animals/bienchen_bluemchen.jpg}} 
\begin{frame}

\end{frame} 
}


%% %% TLIA
{
  \usebackgroundtemplate{\includegraphics[width=1.2\paperwidth]{/home/melanie/Work/pictures/animals/bienchen_bluemchen.jpg}} 
\begin{frame}
\begin{flushright}{\textcolor{white}{Bienchen und Blümchen \dots} }\\[8cm]
\end{flushright}
\pause
\textcolor{white}{\dots aber wissenschaftlich!}
\end{frame} 
}

%% %% Learning Objectives
 
\begin{frame}

 \frametitle{Nach dieser Vorlesung sollten Sie folgendes können}



\begin{block}{Grundlagen:}
\begin{itemize}
\item
 Genetische, hormonelle und psychologische Aspekte der sexuellen Identität erklären
\item
 Häufige primäre und sekundäre Geschlechtsmerkmale benennen
\item
 Bildung, Wirkung, und Regelkreise der Sexualhormone erklären
\item
 die Sexualentwicklung erklären
\item
 Spermienbildung erklären
\item
 den Menstruationszyklus erläutern
\item
Menopause und Klimakterium erklären



\end{itemize}

\end{block}

\begin{block}{Klinik:}
\begin{itemize}
\item
Störungen der Sexualentwicklung benennen und erklären

\item
die Wirkungsweise hormoneller Verhütungsmittel erklären


\end{itemize}

\end{block}


\end{frame}





%% %% %% Main Body



%% - Sexualität: Definitionen
\section{Definitionen}

%% %% Selbst-Test: Grönemeyer
\begin{frame}

\frametitle{Kleiner Selbsttest}


\begin{description}
\item[$\square$]
 Männer kriegen keine Kinder
\item[$\square$]
Männer kriegen dünnes Haar
\item[$\square$]
 Männer sind auch Menschen
\item[$\square$]
 Männer sind etwas sonderbar
\end{description}

$\,$

\end{frame}



\begin{frame}

\frametitle{Kleiner Selbsttest}


\begin{description}
\item[$\checkmark$]
 Männer kriegen keine Kinder
\item[$\square$]
Männer kriegen dünnes Haar
\item[$\checkmark$]
 Männer sind auch Menschen
\item[$\checkmark$]
 Männer sind etwas sonderbar \\[1cm]
\end{description}

\pause

Können wir das besser als Herbert Grönemeyer?

\end{frame}



\begin{frame}
\frametitle{Wann ist ein Mann ein Mann? - Mögliche Antworten}
\pause

\begin{itemize}
\item
Karyotyp (z.B. XY)
\item
Hormonspiegel (z.B. Testosteron)
\item
Primäre Geschlechtsmerkmale (z.B. Penis, Hoden)
\item
Sekundäre Geschlechtsmerkmale (z.B. Bartwuchs)
\item
``Typische'' Körpermerkmale (z.B. Größe, Muskelmasse)
\item
Erfüllen von Geschlechternormen (z.B. Kleidung, Verhaltensweisen)
\item
Selbstwahrnehmung (z.B. Vornamen, Pronomen)
\item
Gesetz (z.B.  Pass)
\end{itemize}

\pause

Diese verschiedenen Definitionen stimmen oft überein, aber oft auch nicht. Einzelheiten können kompliziert sein.

\end{frame}



%% *************
\begin{frame}
\frametitle{Karyotyp}

\begin{columns}[c]

\begin{column}{5cm}



Übliche Definition:  \\46 XX (Frau) oder \\ 46 XY (Mann) \\[0.2 cm]


\end{column}


\begin{column}{5cm}
\begin{center}
  \includegraphics[width=\textwidth]{/home/melanie/Work/pictures/physiology/karyogram.png}
\end{center}

\end{column}



\end{columns}

\pause

(Relativ häufige) Abweichungen:
\begin{itemize}
\item
Androgenresistenz: 46 XY Karyotyp, aber Testosteron hat keine Wirkung, weiblicher Phänotyp
\item
Turner-Syndrom:	Frau: 45 X0 Karyotyp (gonadosomale Monosomie), Ovariendegeneration vor Pubertät
\item
 Klinefelter-Syndrom: Mann: 47 XXY Karyotyp, Hodenunterentwicklung und -funktion
\end{itemize}



\end{frame}




%% %% [right column]
\begin{frame}
\frametitle{Phänotypisches Geschlecht}

\begin{block}{Primäre Geschlechtsmerkmale}

\begin{itemize}
\item
Vulva, Vagina, Uterus,	Ovidukte, Ovarien
\item
Penis, Scrotum, Testes, Prostata, Vas deferens
\end{itemize}

\end{block}

\begin{block}{Sekundäre Geschlechtsmerkmale}

\begin{itemize}
\item
  Schambehaarung, Achselhaare
\item
  Brüste, breite Hüften, schmale Schultern
\item
  männlicher Behaarungstyp, Kehlkopf, ausgeprägte Muskulatur, schmale Hüften, breite Schultern
\end{itemize}
\end{block}


\end{frame}



\begin{frame}
\frametitle{Soziales Geschlecht (``Gender'')}

\begin{itemize}
\item
Geschlechtsidentität (Psychologie)
\item
Soziale Konstruktion
\item
Kann mit dem ``biologischen'' Geschlecht übereinstimmen, muss aber nicht. Biologisches Geschlecht kann (teilweise) angeglichen werden.
\item
Unabhängig von sexueller/romantischer Attraktion
\end{itemize}


\end{frame}



\begin{frame}
\frametitle{Soziales Geschlecht (``Gender'')}

\begin{center}
  \includegraphics[width=0.8\textwidth]{/home/melanie/Work/pictures/physiology/genderunicorn.jpg}
\end{center}


\end{frame}




\section{Sexualhormone}


%% Überblick funktionen, google image search mens magazine
\begin{frame}
\frametitle{Wir erinnern uns:}
\begin{center}
  \includegraphics[width=0.7\textwidth]{/home/melanie/Work/pictures/physiology/hypothalamus_hypophyse_gewebe.jpg}
\end{center}
\end{frame}



%% *************
\begin{frame}
\frametitle{Sexualhormone: Allgemeines}


\begin{block}{Geschlechtsunspezifisch}

Wirkung: Ausschüttung geschlechtsspezifischer Sexualhormone \\[0.5 cm]
 
\pause

GnRH
\begin{itemize} 
\item Peptidhormon 
\item Hypothalamus \(\rightarrow\) Adenohypophyse
\item Wirkung: pulsatile Freisetzung FSH \& LH \\[0.5 cm]
\end{itemize}

\pause

FSH und LH (Gonadotropine) 
\begin{itemize}
\item Glykoproteine 
\item  Adenohypophyse \(\rightarrow\) Gonaden (Hoden/Ovar)
\item Wirkung FSH: Reifung Samenzellen/Eizellen; 
\item Wirkung LH: Reifung Samenzellen/Ovulation und Gelbkörperbildung
\end{itemize}
\end{block}

\end{frame}



\begin{frame}
\frametitle{Geschlechtsspezifische Hormone sind Cholesterolderivate}

\begin{center}
\includegraphics[width=0.7\textwidth]{/home/melanie/Work/pictures/physiology/biosynthese_sexualhormone.png}
\end{center}


\end{frame}
%% %% *************



\begin{frame}
\frametitle{Geschlechtsspezifische Sexualhormone}

\begin{block}{Progesteron}
\begin{itemize}
\item
Bildung in Ovar, Plazenta, Nebennierenrinde
\item
Transport mit CBG
\item
Vorbereitung Genitaltrakt auf und Erhaltung von Schwangerschaft
\item
Niere: \(\uparrow\) Salz- und Wasserausscheidung
\item
ZNS: \(\uparrow\) Körpertemperatur
\item
Progesteronrezeptor PR \(\rightarrow\) Gentranskription
\end{itemize}
\end{block}

\end{frame}

\begin{frame}
 \frametitle{Geschlechtsspezifische Sexualhormone}



\begin{block}{Testosteron (+ Dihydrotestosteron)}
\begin{itemize}
\item
Bildung in Hoden, Ovar, Nebennierenrinde
\item
Transport mit SHBG
\item
 Männliche Geschlechtsdifferenzierung, Spermatogenese, Wachstum und Funktion männlicher Geschlechtsorgane, Libido
\item
Blutbildung
\item
Anabol
\item
ZNS
\item
 Androgenrezeptor AR \(\rightarrow\) Gentranskription
\end{itemize}
\end{block}
\end{frame}

\begin{frame}
 \frametitle{Geschlechtsspezifische Sexualhormone}


\begin{block}{Östradiol (E2)}
\begin{itemize}
\item

Bildung in Ovar, Plazenta, Hoden, Nebennierenrinde
\item
Transport mit SHBG
\item
 Wachstum und Funktion weibliche Geschlechtsorgane
\item
Niere: \(\uparrow\) Salz- und Wasserretention
 \item
 \(\uparrow\) Knochenmasse
\item
 \(\uparrow\) Fetteinlagerung in Unterhaut
\item
 ZNS
\item
ER\(\alpha\) und ER\(\beta\) (Zytoplasma); GPER1 (Plasmamebran)


\end{itemize}
\end{block}

\end{frame}

%% %% *************
%%%%%%%%%%%%%%%%%%%%%%%%%%%
%%%%%%%%% compiled to here
%%%%%%%%%%%%%%%%%%%%%%%%%%%


\section{Sexualentwicklung}

\begin{frame}
\frametitle{Gonaden entwickeln sich im Embryo aus einem gemeinsamen Vorläufer}

\begin{center}
\includegraphics<1>[width=0.6\textwidth]{/home/melanie/Work/pictures/physiology/gonad_differentiation_A.jpg}

\includegraphics<2>[width=0.6\textwidth]{/home/melanie/Work/pictures/physiology/gonad_differentiation_B.jpg}

\includegraphics<3>[width=0.6\textwidth]{/home/melanie/Work/pictures/physiology/gonad_differentiation.jpg}
\end{center}

\end{frame}


\begin{frame}
%% [Right column]

\frametitle{Pubertät}
Sexuelle Reifung

\begin{block}{Mädchen}
\begin{itemize}
\item
Beginn mit 8-13 Jahren
\item
Thelarche, Pubarche, Wachstumsschub, Menarche
\item
Oestrogen: sekundäre Geschlechtsmerkmale
\end{itemize}
\end{block} 

\begin{block}{Jungen:}
\begin{itemize}
\item
Beginn mit 9-14 Jahren
\item    
Hodewachstum, Pubarche, Spermarche
\item
Testosteron: sekundäre Geschlechtsmerkmale
\end{itemize}
\end{block}

\pause

Pubertätsblocker: GnRH-Analoga, chronisch verabreicht. \\
\textcolor{theme}{Funktionieren wie? Und warum braucht man sie?}

\end{frame}


\begin{frame}
%% [Right column]

\frametitle{Pubertät}



\begin{block}{Pubertätsblocker:} 
GnRH-Analoga, chronisch verabreicht. \\

Zunächst Erhöhung von FSH, LH (und daher auch Testosteron/Östrogen), aber (nach ca. 2 Wochen) Herunterregulierung der GnRH-Rezeptoren in der Hypophyse (reversibel)


Einsatz z.B. bei Pubertas praecox oder bei trans Kindern, um die Pubertät hinauszuzögern.

\end{block}

\end{frame}

%%%%%%%%%%%%%%%%%%%%%%%%%%%
%%%%%%%%% compiled to here
%%%%%%%%%%%%%%%%%%%%%%%%%%%


 \section{Spermienbildung}


\begin{frame}
\frametitle{Spermienbildung}



\begin{center}
\includegraphics[width=0.8\textwidth]{/home/melanie/Work/pictures/physiology/Spermatogenese.png}
\end{center}

\end{frame}


\begin{frame}
\frametitle{Spermienbildung}


\begin{center}
\includegraphics[width=\textwidth]{/home/melanie/Work/pictures/physiology/hormonelle-regulierung-spermatogenese.png}

\end{center}


\end{frame}



\begin{frame}
\frametitle{Spermienbildung}

\begin{itemize}
\item
Pulsatile Freisetzung von GnRH aus dem Hypothalamus
\item
Freisetzung von LH und FSH aus der Hypophyse
\pause
\item
Im Kernepithel der Hoden: FSH stimuliert Sertoli-Zellen und fördert die Reifung der Gameten
\item
Im Interstitium der Hoden: LH stimuliert Leydig-Zellen und bewirkt Freisetzung von Testosteron
\pause
\item
Testosteron und Inhibin koppeln negativ an die Hypophyse zurück
\pause
\item
Finale Reifung der Spermien in den Nebenhoden

\end{itemize}

\pause

\begin{center}
\includegraphics[width=\textwidth]{/home/melanie/Work/pictures/physiology/spermium.png}
\end{center}


\end{frame}






\section{Menstruationszyklus}

\begin{frame}
\frametitle{Menstruationszyklus}

\begin{columns}[c]

\begin{column}{4cm}

\begin{center}
\includegraphics[width=\textwidth]{/home/melanie/Work/pictures/physiology/hypothalamus_hypophyse_ovar.png}
\end{center}

\end{column}

\begin{column}{7cm}

\pause

\begin{block}{Aufgaben Zyklus:}
\begin{itemize}
\item
 
 Bereitstellung einer (einzelnen) befruchtungsfähigen Eizelle
\item
 Vorbereitung der Uterusschleimhaut (Endometrium) auf Implantation (Einnistung) des Embryos (Blastocysten)
\item

 Erleichterung der Spermienaufnahme

\end{itemize}
\end{block}


\pause

\begin{block}{Phasen}
\begin{itemize}
\item
Follikelphase: 14 Tage (7-21)
\item 
Ovulationsphase: 15 Minuten 
\item
Lutealphase: 14 Tage

\end{itemize}
\end{block}




\end{column}


\end{columns}

\end{frame}

%% %% *************



\begin{frame}
\frametitle{Der Follikel im Zyklus - Follikelphase}


\begin{columns}[c]

\begin{column}{4cm}

\begin{center}
\includegraphics[width=\textwidth]{/home/melanie/Work/pictures/physiology/hypothalamus_hypophyse_ovar.png}
\end{center}

\end{column}

\begin{column}{7cm}

\begin{itemize}
\item
Pulsatile Freisetzung von GnRH, stimuliert die Freisetzung von FSH und LH in die Zirkulation
\item
FSH und LH erreichen den im Ovar befindlichen Follikel erreicht.
\item
FSH und LH binden an Rezeptoren auf Granulosa- und Thekazellen
\item
 FSH und LH rekrutieren Kohorte antraler Follikel (10-20) zur weiteren Entwicklung
\item
 Einer dieser rekrutierten Follikel wird der dominante Follikel werden, welcher die befruchtungsfähige Eizelle bereitstellen wird
\end{itemize}

\end{column}

\end{columns}

\end{frame}



%% %% *************

\begin{frame}
\frametitle{ Der Follikel im Zyklus - Follikelphase}

\begin{center}
\includegraphics[width=\textwidth]{/home/melanie/Work/pictures/physiology/Follikel_FSH_LH.png}
\end{center}

\end{frame}


\begin{frame}
\frametitle{ Der Follikel im Zyklus - Follikelphase}

\begin{columns}[c]
\begin{column}{5cm}
FSH an Granulosazellen:
\begin{itemize}
\item
Stimuliert Proliferation 
\item 
Produktion von Aromatase. 
\item
Ausbildung von LH und FSH Rezeptoren auf den Granulosazellen
\end{itemize}

\end{column}

\begin{column}{5cm}
LH an Thekazellen:
\begin{itemize}
\item
Vermehrte LH Rezeptoren auf Thekazellen, 
\item 
Vermehrte Aufnahme von Cholesterol aus dem Blut
\item
verstärkte Umwandlung in Androgene, Androgene diffundieren in die Granulosazellen (wo Aromatase vorhanden ist)
\end {itemize}
\end{column}


\end{columns}

$\,$\\[0.5 cm]


Dominanter Follikel: Größter, wächst am schnellsten, höchste Östrogenproduktion.


\end{frame}



%% %% *************

%% %% Der Follikel im Zyklus - Follikelphase


%% %% [left side]
%% %% [Same Hypothalamus-Hypophyse-Ovar Abbildung,  aber negatives Feedback von Östrogen/Inhibin zurück zu Hypothalamus und Hypophyse]


%% %% [right side]

%% %% Hohe Östrogenproduktion durch dominanten Follikel:

%% %% -> Negatives Feedback auf FSH- und LH-Freisetzung

%% %% -> Inhibiertes Wachstum der anderen Follikel -> Atresie

%% %% <- Domimamter Follikel ist hochsensitiv udn braucht nur basale [FSH] und [LH]

%% %% Dominanter Follikel = einziger Follikel für Ovulation

%% %% -> [Östrogen] im Blut > 200 pg/ml > 48h


%% %% *************

%% %% Der Follikel im Zyklus - Follikelphase, Ovulation

%% %% [left side]
%% %% [Same Hypothalamus-Hypophyse-Ovar Abbildung,  aber jetzt positives Feedback von Östrogen zurück zu Hypothalamus und Hypophyse]

%% %% [right side]

%% %% Hoher Anstieg [LH]:

%% %% a) Collagenasen -> lokal reduzierte Integrität Follikelwand

%% %% b) Anstieg des intrafollikulären Flüssigkeitsdrucks

%% %% -> Eizelle tritt aus Follikel heraus 

%% %% [Photo]

%% %% *************

%% %% Der Follikel im Zyklus - Lutealphase


%% %% [left side]
%% %% [Same Hypothalamus-Hypophyse-Ovar Abbildung,  CL im Ovar -> Östrogen, Progesteron, Negatives Feebdack zu Hypothalamus und Hypophyse]

%% %% Hohe [LH] im Ovar:
%% %% 1. Theka- und Granulosazellen wandeln sich in Lutealzellen um -> Corpus luteum = Gelbkörper

%% %% 2. Veränderung in Aktivität steroidogener Enzyme -> Progesteron-Synthese und Freisetzung vom Corpus luteum dominiert (Östrogen)

%% %% -> Reduktion FSH- und LH-Freisetzung (zusammen mit Östrogen)

%% %% Entweder:
%% %% - Fertilisierung und Implantation haben stattgefunden
%% %%   - Humanchoriogonadotropin (HCG)-Sekretion Conceptus
%% %%   - Stimulierung Lutealzellen
%% %%   - Progesteron wird weiter produziert

%% %% Oder:
%% %% - Keine Fertilisierung oder Implantation
%% %%   - Keine HCG-Sekretion
%% %%   - Rückbildung des Coropus Luteum (Luteolyse)
%% %%   - Corpus albicans -> kein Progesteron + Östrogen
%% %%   - Inhibierung der GnRH- udn FSH/LH-Freisetzung fällt weg
%% %%   - Ein neuer Zyklus beginnt

%% %% *************

%% %% Der Uterus im Zyklus

%% %% - Zyklischer Auf- und Abbau des Endometriums (Stratum functionale).

%% %% [Abbildung Endometrium während verschiedener Phasen]

%% %% [Aufbau: 2: Profiferationsphase ]
%% %% [Abbildung Follikel im Ovar]

%% %% - \uparrow [Östrogen] (wachsende Follikel)
%% %%   - -> Proliferation des Endometriums
%% %%   - -> Entwicklung von Spiralarterien


%% %% [Aufbau: 3: Sekretionsphase ]
%% %% [Abbildung CL im Ovar]

%% %% - \uparrow [Progesteron] (und [Östrogen]) (Corpus luteum):
%% %%   - -> Weiterausbildung Endometrium + Spiralarterien + Drüsen
%% %%   - -> Schleimsekretion von Drüsen

%% %% Erfolgreiche Implantation: Endometriale Zellen wandeln sich in Dezidualzellen um -> Teil der Plazenta


%% %% [Aufbau: 3: Menstruation ]

%% %% [Abbildung CA im Ovar]

%% %% - Wenn keine Einnistung stattfindet:
%% %% - Luteolyse bewirkt Reduktion der zirkulierenden [Östrogen] und [Progesteron] -> Konstriktion der Spiralarterien -> Ischämie -> Entzündungsreaktion -> Abbau und Desquamation = Menstruation

%% %% *************

%% %% Die Zervix im Zyklus
 

%% %% [Abbildung: Zervix im Zyklus]


%% %% *************

%% %% [left side:]

%% %% Follikelphase:
%% %% - Abbau Endometrium des vorherigen Zyklus (Menstruation)
%% %% - Ansteigende [FSH] und [LH] -> Rekrutierung Follikelkohorte, aus der dominanter Follikel hervorgeht -> Östrogen inhibiert FSH- und LH- Freisetzung und unterstützt Proliferation Endometriums und Spiralarterienausbildung
%% %% - Wenn kritische [Östrogen] erreicht ist, werden hohe [LH] freigesetzt ->

%% %% Ovulation:
%% %% - Freisetzung befruchtungsfähiger Eizelle
%% %% - Zervikalkanal und Zervixschleim aufnahmebereit

%% %% Lutealphase:
%% %% - Corpus luteum produziert hohe [Progesteron] (und [Östrogen]) -> FSH- und LH- Freisetzung wird inhibiert, Endometrium weiter aufgebaut
%% %% - Keine Schwangerschaft: Corpus luteum degeneriert -> [Progesteron] und [Östrogen] sinken -> [FSH] und [LH] steigen an; Ischämie Endometrium

%% %% [left side]

%% %% [Abbildung Hormonkonzentrationen in Ovar und Uterus, Follikel, Endometrium während des Zyklus]


\section{Hormonelle Verhütung}

\begin{frame}
\frametitle{Hormonelle Verhütung: Wo würden Sie ansetzen?}

\begin{center}
\includegraphics[width=0.5\textwidth]{/home/melanie/Work/pictures/physiology/zyklus.png}
\end{center}
\end{frame}



{
  \usebackgroundtemplate{\includegraphics[width=1.2\paperwidth]{/home/melanie/Work/pictures/physiology/pill.jpg}} 
\begin{frame}
\frametitle{Hormonelle Verhütung: Verhütungspille}

Künstliche Analoga zu Östrogen-Progesteron: \\
\begin{itemize}
\item
Negatives Feedback verhindert Ausschüttung von LH und FSH \\
\item
Ovulation wird gehemmt \\
\item
Cervixschleim bleibt dickflüssig \\[0.2 cm]
\end{itemize}

\pause

Pearl-Index (ideale Anwendung): 0.3 (Schwangerschaften bei 100 Anwender*innen innerhalb eines Jahres)\\
Pearl-Index (tatsächlich): 1-8 \\

Nebenwirkungen: Erhöhtes Thromboserisiko\\[4cm]





\end{frame}
}



\begin{frame}
\frametitle{Hormonelle Verhütung: Alternativen}

\begin{itemize}
\item
Andere Zusammensetzung (nur Gestagen): Vermeidet Nebenwirkungen von Östrogen:
\begin{itemize}
\item
Minipille
\item
Verhütungsstäbchen
\end{itemize}
\pause
\item
Andere Darreichungsform: Verringert das Risiko von Anwendungsfehlern:
\begin{itemize}
\item
Vaginalring
\item
Hormonpflaster
\item
Verhütungsstäbchen
\end{itemize}

\end{itemize}

\pause
``Pille danach'':  Notfall-Kontrazeption (am besten innerhalb von 24 Stunden) mit Gestagen oder Ulipristalacetat (Progesteronrezeptor-Modulator): Hemmt Ovulation und Einnistung



\end{frame}

\section{Menopause}




%% Menopause
\begin{frame}
\frametitle{Menopause}



Menopause = Letzte Menstruationsblutung

Klimakterium (Wechsel):
\begin{itemize}
\item
Zeit um die Menopause (typischerweise 45-55 Jahre)
\item
Erschöpfung der ovariellen Reserve 
\item
Reduktion der Östrogenproduktion
\item
 Oligomenorrhoe
\item
 Anovulatorische Zyklen nehmen zu
\item 
Erhöhter Gonadotropinspiegel

\end{itemize}

\end{frame}



%% %% Menopause
\begin{frame}
\frametitle{Menopause}

\begin{columns}[c]

\begin{column}{7cm}

\begin{center}
 \includegraphics[width=\textwidth]{/home/melanie/Work/pictures/physiology/menopause.png}
\end{center}



\end{column}

\pause

\begin{column}{4cm}

Behandlung: \\[0.2 cm]


Linderung einzelner Symptome \\
Hormonersatztherapie


\end{column}


\end{columns}



 \end{frame}




%% %% %% %% Review


\begin{frame}

 \frametitle{Jetzt* sollten Sie folgendes können}



\begin{block}{Grundlagen:}
\begin{itemize}
\item
 Genetische, hormonelle und psychologische Aspekte der sexuellen Identität erklären
\item
 Häufige primäre und sekundäre Geschlechtsmerkmale benennen
\item
 Bildung, Wirkung, und Regelkreise der Sexualhormone erklären
\item
 die Sexualentwicklung erklären
\item
 Spermienbildung erklären
\item
 den Menstruationszyklus erläutern
\item
Menopause und Klimakterium erklären



\end{itemize}

\end{block}

\begin{block}{Klinik:}
\begin{itemize}
\item
Störungen der Sexualentwicklung benennen und erklären
\item
die Wirkungsweise hormoneller Verhütungsmittel erklären


\end{itemize}

\end{block}


\end{frame}




%% %% %% %% Feedbackhinweisblock

\begin{frame}
\frametitle{Danke für Ihr Feedback!}

\begin{columns}[c]

\begin{column}{6cm}
\begin{center}
 \includegraphics[width=\textwidth]{/home/melanie/Work/pictures/metaphore/smilie_balloons.jpg}
\end{center}

\end{column}

\begin{column}{4cm}


\begin{center}
\includegraphics[width=\textwidth]{feedback_QR.png}
\end{center}
\end{column}


\end{columns}

\end{frame}



%% %% %% Bildnachweis
\begin{frame}
\frametitle{Bildnachweis}

\begin{tiny}

Teile dieser Vorlesung wurden übernommen von einer Vorlesung von Prof. Maike Glitsch, Medical School Hamburg. Herzlichen Dank!


 
\begin{itemize}

   

\item
Bienchen und Blümchen. Photo by \href{https://unsplash.com/@stereophototyp?utm_source=unsplash&utm_medium=referral&utm_content=creditCopyText}{Sara Kurfeß} on \href{https://unsplash.com/s/photos/flowers-and-bees?utm_source=unsplash&utm_medium=referral&utm_content=creditCopyText}{Unsplash}

\item
Biosynthese der Sexualhormone. Modifiziert nach einer Version Von User:Mikael Häggström derivative work (german translation): Benff - File:Steroidogenesis.svg, CC BY-SA 3.0, \url{https://commons.wikimedia.org/w/index.php?curid=103474201}


\item

Entwicklung der Gonaden aus einer bipotentialen Vorgänger-Gonade: Von Teixeira, J., Rueda, B.R., and Pru, J.K., Uterine Stem cells (September 30, 2008), StemBook, ed. The Stem Cell Research Community, StemBook, doi/10.3824/stembook.1.16.1, \url{http://www.stembook.org}. - [1] DirectStemBook Figure 2 The uterus differentiates from the fetal Müllerian ducts.Teixeira, J., Rueda, B.R., and Pru, J.K., Uterine Stem cells (September 30, 2008), StemBook, ed. The Stem Cell Research Community, StemBook, doi/10.3824/stembook.1.16.1, \url{http://www.stembook.org.}, CC BY 3.0, \url{https://commons.wikimedia.org/w/index.php?curid=25463769}

\item
Gender-Unicorn. Trans Student Educational Resources, 2015. ``The Gender Unicorn.'' \url{http://www.transstudent.org/gender.}, Creative Commons.



\item
Hypothalamus-Hypophyse-Ovar Achse. Aus:    Sabine Segerer, Barbara Sonntag, Kai Gutensohn \& Christoph Keck (2018). Hormonanalytik – was der Frauenarzt wissen muss. \emph{Der Gynäkologe} volume 51, pages 891–909.

\item
Karyogramm. Von Courtesy: National Human Genome Research Institute - Found on :National Human Genome Research (USA)This image was copied from wikipedia:en., Gemeinfrei, \url{https://commons.wikimedia.org/w/index.php?curid=7853183}


%% all lectures
\item
Luftballons mit frohen und traurigen Smilies. Photo by \href{https://unsplash.com/@artbyhybrid?utm_source=unsplash&utm_medium=referral&utm_content=creditCopyText}{Hybrid} on \href{https://unsplash.com/s/photos/feedback?utm_source=unsplash&utm_medium=referral&utm_content=creditCopyText}{Unsplash}
%%%%%%%%%%%

\item
Signalwege Hypothalamus - Hypophysenvorderlappen - Gewebe - Funktion.  Aus: Schöfl C. Neuroendokrine Dysfunktion nach Schädelhirntrauma und Subarachnoidalblutung. \emph{Blickpunkt der Mann} 2008; 6 (Sonderheft 1): 22-24

\item
Spermatogenese. Von Ebricca, CC BY-SA 3.0, \url{https://commons.wikimedia.org/w/index.php?curid=23864683}

\item
Spermium. Public Domain, \url{https://commons.wikimedia.org/w/index.php?curid=1713008}

\item
Symptome der Menopause. Mikael Häggström, CC0, via Wikimedia Commons
\end{itemize}
\end{tiny}
\end{frame}




\begin{frame}
\frametitle{Bildnachweis}

\begin{tiny}


\begin{itemize}
\item
Übersicht Menstruationszyklus. Von Thomas Steiner - mainly Image:MenstrualCycle.png and some infos from the internet, CC BY-SA 2.5,\url{ https://commons.wikimedia.org/w/index.php?curid=1676656}

\item
Verhütungspille. Photo by \href{https://unsplash.com/@rhsupplies?utm_source=unsplash&utm_medium=referral&utm_content=creditCopyText}{Reproductive Health Supplies Coalition} on \href{https://unsplash.com/s/photos/contraceptive?utm_source=unsplash&utm_medium=referral&utm_content=creditCopyText}{Unsplash}

\item
Wirkung von FSH und LH auf Hoden. Aus:  Dr. Med. Elena Santiago Romero (gynäkologin), Dr. Med. Mark P. Trolice (gynäkologe), Zaira Salvador (embryologin) Und Romina Packan (invitra staff).  Männliche und weibliche Sexualhormone- welche Funktionen haben sie? Aktualisiert am 27/08/2019. \url{https://www.invitra.de/sexualhormone/}


\item
Wirkung von FSH und LH auf Zellen des Follikels. Aus:  Dr. Med. Elena Santiago Romero (gynäkologin), Dr. Med. Mark P. Trolice (gynäkologe), Zaira Salvador (embryologin) Und Romina Packan (invitra staff).  Männliche und weibliche Sexualhormone- welche Funktionen haben sie? Aktualisiert am 27/08/2019. \url{https://www.invitra.de/sexualhormone/}

\end{itemize}
\end{tiny}
\end{frame}





\end{document}

%%% Frequently used snippets

%% \begin{columns}[c]

%% \begin{column}{5cm}
%% \end{column}

%% \begin{column}{5cm}
%% \end{column}


%% \end{columns}
